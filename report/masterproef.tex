\documentclass[master=elt,masteroption=eg,english]{kulemt}
\setup{% Verwijder de "%" op de volgende lijn bij UTF-8 karakterencodering
  %inputenc=utf8,
  title={Model identification and learning control in autonomous driving},
  author={Stijn Staring},
  promotor={Prof.\,dr.\,ir.\ Jan Swevers},
  assessor={Prof.\,dr.\,ir.\ Bert Pluymers\and Prof.\,dr.\,ir.\ Herman Bruyninckx},
  assistant={dr.\,ir.\ Son Tong}}
% DENK ERAAN OM DE MASTER AAN TE PASSEN OP HET EINDE!!
% Verwijder de "%" op de volgende lijn als je de kaft wil afdrukken
%\setup{coverpageonly}
% Verwijder de "%" op de volgende lijn als je enkel de eerste pagina's wil
% afdrukken en de rest bv. via Word aanmaken.
%\setup{frontpagesonly}

% Kies de fonts voor de gewone tekst, bv. Latin Modern
\setup{font=lm}

% Hier kun je dan nog andere pakketten laden of eigen definities voorzien

% Tenslotte wordt hyperref gebruikt voor pdf bestanden.
% Dit mag verwijderd worden voor de af te drukken versie.
\usepackage[pdfusetitle,colorlinks,plainpages=false]{hyperref}

%%%%%%%
% Om wat tekst te genereren wordt hier het lipsum pakket gebruikt.
% Bij een echte masterproef heb je dit natuurlijk nooit nodig!
\IfFileExists{lipsum.sty}%
 {\usepackage{lipsum}\setlipsumdefault{11-13}}%
 {\newcommand{\lipsum}[1][11-13]{\par Hier komt wat tekst: lipsum ##1.\par}}
%%%%%%%

%\includeonly{chap-n}
\begin{document}

\begin{preface}
	I would like to thank my family in the first place. During the history of my studies they always have been my biggest fans and I want to show my gratitude for the opportunities they have given me. I also want to thank my promoter Professor Swevers at the KU Leuven and Dr. Tong my mentor at Siemens for the professional discussions and tips they have given me in order to improve results.

\end{preface}

\tableofcontents*

\begin{abstract}
  The \texttt{abstract} environment contains a more extensive overview of
  the work. But it should be limited to one page.

  \lipsum[1]
\end{abstract}

\begin{abstract*}
  In dit \texttt{abstract} environment wordt een al dan niet uitgebreide
  Nederlandse samenvatting van het werk gegeven.
  Wanneer de tekst voor een Nederlandstalige master in het Engels wordt
  geschreven, wordt hier normaal een uitgebreide samenvatting verwacht,
  bijvoorbeeld een tiental bladzijden. 

  \lipsum[1]
\end{abstract*}

% Een lijst van figuren en tabellen is optioneel
%\listoffigures
%\listoftables
% Bij een beperkt aantal figuren en tabellen gebruik je liever het volgende:
\listoffiguresandtables
% De lijst van symbolen is eveneens optioneel.
% Deze lijst moet wel manueel aangemaakt worden, bv. als volgt:
\chapter{List of Abbreviations and Symbols}
\section*{Abbreviations}
\begin{flushleft}
  \renewcommand{\arraystretch}{1.1}
  \begin{tabularx}{\textwidth}{@{}p{12mm}X@{}}
    LoG   & Laplacian-of-Gaussian \\
    MSE   & Mean Square error \\
    PSNR  & Peak Signal-to-Noise ratio \\
  \end{tabularx}
\end{flushleft}
\section*{Symbols}
\begin{flushleft}
  \renewcommand{\arraystretch}{1.1}
  \begin{tabularx}{\textwidth}{@{}p{12mm}X@{}}
    42    & ``The Answer to the Ultimate Question of Life, the Universe,
            and Everything'' according to \cite{h2g2} \\
    $c$   & Speed of light \\
    $E$   & Energy \\
    $m$   & Mass \\
    $\pi$ & The number pi \\
  \end{tabularx}
\end{flushleft}

% Nu begint de eigenlijke tekst
\mainmatter

\chapter{Introduction}
\label{cha:intro}

Zoek hier een bron die een quote maakt over zelfrijdende wagens. Inspiratie in thesissen en taak voertuigsystemen (iemand die er veel van weet)

\section{Wat moet er vermeld worden}
Goed refereren naar de bronnen waar over spreekt. Bespreek de vooruitgangen die gemaakt zijn in het gebruik van zelf rijden wagens en probeer te verkopen wrm het nu nuttig is om naast veiligheid te kijken naar comfort. 

wagenziektes verminderen
een klant koopt een auto waar hij zich goed in voelt --> comfort is specifiek
comfort = vertrouwen




(meer op het einde van de introductie)
Dit hoofdstuk geeft vooral een overzicht over de theorie en de opbouw van het probleem.
Wat zijn de doelen van het project --> zie slides / wat is de probleemstelling.
Wat is de kapstok --> drie stappen learning, planning, controlling, validatie
Hoe zal het er in grote lijnen uit gaan zien en welke softwares zijn er allemaal aan te pas gekomen
Welke voorzieningen heeft Siemens hieromtrent.

Wat is een optimal control problem en wat is model predictive control? (zie paper VS)
--> wat dieper ingaan op wat ik geleerd heb bij het vak optimization --> zie ook important thesis notes van optimization

Maak een overview van het control sequence --> flowchart zoals op p5 zwitser. Maak zo veel mogelijk om de stromingen van de stappen bijvoordbeeld 3 stappen in project goed weer te geven. Veel figuren!


%%% Local Variables: 
%%% mode: latex
%%% TeX-master: "thesis"
%%% End: 

\chapter{Comfort definition and modelling}
\label{cha:1}

Dan komt de vraag wat is comfort precies? Literatuur studie...
Waarnaar kan men kijken als men het over comfort heeft. 
Lane change bekeken om comfort te valideren --> zeg dat er geen iso standaarden aanwezig zijn.
Hoe wordt een bestuurder gemoddelleerd --> dit wordt gedaan door een kansverdeling.
Waarom is entropie nuttig om deze bestuurder te kunnen bekijken? Doe hier meer ondezoek over en verantwoord het gebruik hiervan. Conclusie komt af met comfort parameters die verder worden gebruikt als features.
Bij de uitleg van de features en waarom er versnelling en acceleratie wordt gebruikt, basseer ook op paper 7 van hoofdpaper

Wat wordt er in de literatuur al gebruikt om comfort te modelleren en geef een overzicht.

Leg uit hoe komt aan entropy distribution komt --> kan beroep doen op ref 2 en 20 van het hoofdrapport (both are assuming an exponential distribution) (IMPORTANT)


Dit is de reden voor het gebruik van de maximum entropie distributie: 
	To learn observed behavior, we aim to model the distribution
	that underlies the empirical sample trajectories.
	Following Abbeel and Ng [1], we aim to find a model that
	induces distributions that match, in expectation, the feature
	values fD of the empirical trajectories D, yielding
	Ep(x)[f (x)] = fD =
	1
	jDj
	X
	xk2D
	f (xk): (1)
	In general, however, there is not a unique distribution that
	matches the features. Ziebart et al. [24] resolve this ambiguity
	by applying the principle of maximum entropy [10], which
	states that the distribution with the highest entropy represents
	the given information best since it does not favor any
	particular outcome besides the observed constraints. (this is the least baised distribution)
	Modelling expected featueres by hybrid monte carlo method: https://reader.elsevier.com/reader/sd/pii/037026938791197X?token=71567B7640C402F2FF578E34E3BB7914CA14E4A1A29DE88D9534D96C9305E13DA52D0424FF1475A822FCE784725196D3
%	ref: C:\Users\t2vosx\OneDrive\Documenten\Leuven\Thesis\References2\Citations on main %paper\Learning to Predict Trajectories of Cooperatively Navigating Agents.pdf>

 	paper: Feature-based prediction of trajectories for socially compliant navigation
	which
	states that the distribution with the highest entropy represents
	the given information best since it does not favor any
	particular outcome besides the observed constraints.
	
%	• Weet dat exponentiële vorm oplossing is van boven staand optimizatie probleem. zie foto
%	--> volgt uit FONC --> oplossing moet (non convex) voldoen aan KKT conditions. Er zijn enkel maar equality constraints aanwezig --> primal feasibility en lagrange stationarity moeten voldaan worden --> LS wordt gecontroleerd drm van de Euler - lagrange vergelijking te gebruiken. 



\section{The First Topic of the Chapter}
First comes the introduction to this topic.


\subsection{An item}
Please don't abuse enumerations: short enumerations shouldn't use
``\verb|itemize|'' or ``\texttt{enumerate}'' environments.
So \emph{never write}: 
\begin{quote}
  The Eiffel tower has three floors:
  \begin{itemize}
  \item the first one;
  \item the second one;
  \item the third one.
  \end{itemize}
\end{quote}
But write:
\begin{quote}
  The Eiffel tower has three floors: the first one, the second one, and the
  third one.
\end{quote}

\section{A Second Topic}


\subsection{Another item}


\section{Conclusion}
The final section of the chapter gives an overview of the important results
of this chapter. This implies that the introductory chapter and the
concluding chapter don't need a conclusion.



%%% Local Variables: 
%%% mode: latex
%%% TeX-master: "thesis"
%%% End: 

\chapter{The Learning Algorithm\\}
\label{cha:2}
Herinner de lezer nog even de structuur die gaat worden gevolgd. 
Learning, planning, tracking, validatie.
Dit hoofdstuk zal over het learning gaan.
Hoe is algorithm opgebouwd? Wrm wordt dit zo gedaan?
Welke vehicle modellen wordt er gebruikt? Wrm mag men hier een simple vehicle mode gebruiken?
Dit is gemachtigd omdat men hier de omgeving wil scannen voor een feasible pad --> dit wordt trager gedaan dan de tracking.(tracking zal gebruik maken van een meer complex model) Path planning ligt focus vooral op de omgeving.
Goed refereren naar het rapport en VS rapport
Wat zijn de assumpties die werden genomen?

Hoe zal de methode gevalideerd worden? Leg de twee methodes uit: code generatie en kijken of de wegings factoren terug gevonden kunnen worden? Mappen de feature values met de values van het geobserveerde pad? --> is het doel dat gevolgd probeert te worden haalbaar? 

Ga hier niet meer te diep in op de entropie. Leg het hier meer intuitief uit om de lezer niet te verwaren. 

Vermeld afleiding van algortihm. Leg uit in Thesis hoe komt aan gradient die gebruikt. Zie papers: Ziebart et al and Kretzschmar et al.

Modeleer een andere bestuurder. Can try to reproduce a data set with a change of parameters which represents a different driver. Can check that the learned model is also different. Hiermee aantonen dat er ook echt andere wegingsfactoren worden gegenereerd en dat de specifieke driving characteristics worden meegenomen.

Ligt een tipje van de sluier op : hoe zal de data gegenereerd worden? (dit moet kort blijven)
Plot simulink model en duidt de blokken aan die zullen worden ingevuld. Hier gaat dieper in gegaan worden in de volgende hoofdstukken. 

Maak een vermelding dat men het menselijke gedrag van het geleerde model kan nagaan met een Turing test.\\

Maak een plotje zoals paper Learning to Predict Trajectories of Cooperatively Navigating Agents --> feature variance afwijking en average error. (zelfde plotjes als al de papers)\\



Check uitgebreide samenvatting van hoofdpaper op oneNote.

Schrijf een paragraaf over de theta update --> zie RPROP methode --> beschrijf wrm beter is dan andere methodes die gezien werden. Bespreek hoe de parameters werden gekozen. 


Kan vermelding maken dat in deze thesis de features zijn gekozen met de hand --> men kan proberen om de features ook te leren van date (Characterizing Driving Styles with Deep Learning)

\clearpage


%Afleiding van exponentiël functie zie paper: Feature-based prediction of trajectories for socially compliant navigation (foto) --> weights are lagrange coefficients.


\section{The First Topic of this Chapter}


\subsection{An item}
A master's thesis is never an isolated work. This means that your text must
contain references. On-line documents\cite{wiki} as well as
books\cite{pratchett06:_good_omens} can be referenced.

\section{Figures}
Figures are used to add illustrations to the text. The \fref{fig:logo} shows
the KU~Leuven logo as an illustration.
\begin{figure}
  \centering
  \includegraphics{logokul}
  \caption{The KU~Leuven logo.}
  \label{fig:logo}
\end{figure}

\section{Tables}
Tables are used to present data neatly arranged. A table is normally
not a spreadsheet! Compare \tref{tab:wrong} en \tref{tab:ok}: which table do
you prefer?

\begin{table}
  \centering
  \begin{tabular}{||l|lr||} \hline
    gnats     & gram      & \$13.65 \\ \cline{2-3}
              & each      & .01 \\ \hline
    gnu       & stuffed   & 92.50 \\ \cline{1-1} \cline{3-3}
    emu       &           & 33.33 \\ \hline
    armadillo & frozen    & 8.99 \\ \hline
  \end{tabular}
  \caption{A table with the wrong layout.}
  \label{tab:wrong}
\end{table}

\begin{table}
  \centering
  \begin{tabular}{@{}llr@{}} \toprule
    \multicolumn{2}{c}{Item} \\ \cmidrule(r){1-2}
    Animal    & Description & Price (\$)\\ \midrule
    Gnat      & per gram    & 13.65 \\
              & each        & 0.01 \\
    Gnu       & stuffed     & 92.50 \\
    Emu       & stuffed     & 33.33 \\
    Armadillo & frozen      & 8.99 \\ \bottomrule
  \end{tabular}
  \caption{A table with the correct layout.}
  \label{tab:ok}
\end{table}


\section{Conclusion}
The final section of the chapter gives an overview of the important results
of this chapter. This implies that the introductory chapter and the
concluding chapter don't need a conclusion.



%%% Local Variables: 
%%% mode: latex
%%% TeX-master: "thesis"
%%% End: 

\chapter{Path Planning MPC}
\label{cha:3}

MPC --> path planning met het gevonden model.

Ga hier volledig in op wat MPC is.
Valideer de MPC code door na te gaan wat de invloed is bij het varieren van de gevonden parameters.
Hoeveel zal het verschillen? Ga in op het gebruikte point model en bespreek de gelijkenissen en de verschillen van het learning algorithm. (zie ook notes VS en verbeteringen NAETS)




\section{The First Topic of the Chapter}
First comes the introduction to this topic.



\subsection{An item}
Please don't abuse enumerations: short enumerations shouldn't use
``\verb|itemize|'' or ``\texttt{enumerate}'' environments.
So \emph{never write}: 
\begin{quote}
	The Eiffel tower has three floors:
	\begin{itemize}
		\item the first one;
		\item the second one;
		\item the third one.
	\end{itemize}
\end{quote}
But write:
\begin{quote}
	The Eiffel tower has three floors: the first one, the second one, and the
	third one.
\end{quote}

\section{A Second Topic}


\subsection{Another item}


\section{Conclusion}
The final section of the chapter gives an overview of the important results
of this chapter. This implies that the introductory chapter and the
concluding chapter don't need a conclusion.

\chapter{Path tracking MPC}
\label{cha:4}

MPC --> path tracking met non lin model.
Bespreek non lin model
Bespreek resultaten.
Valideer de resultaten. Wat gebeurd er als de parameteres iets anders worden geschat? Zeker de parameters voor de banden zijn moeilijk om te schatten. Bespreek de tekort komingen van de het model en valideer het model hoe de referencie wordt getrackt met andere parameter values. hoe robuust is de mpc?
(Zei paper VS)\\

Hier kan men praten over de implementatie in het simulink model. Het zou goed zijn om het ACADO model te vervangen door een tracking MPC die in CasADi geschreven is. (reference path to follow wordt niet geupdated maar blijft constant in model --> dit is niet hoe het werkt in de realiteit.) Kan als echt tegoei wilt doen ook inladen in de template die gekregen heb van Flavia --> PID vervangen door MPC en path planner. Een simulink model is nodig om er het 15 dof vehicle model in te kunnen verwerken. \\



\section{The First Topic of this Chapter}
\subsection{Item 1}
\subsubsection{Sub-item 1}


\subsubsection{Sub-item 2}


\subsection{Item 2}


\section{The Second Topic}


\section{Conclusion}

%%% Local Variables: 
%%% mode: latex
%%% TeX-master: "thesis"
%%% End: 

\chapter{Validatie}
\label{cha:5}

Bespreek de validatie van de methode. Implementeer similaties in prescan.
Bespreek de verschillende software tools bij Siemens --> Amesim, simulink, prescan.
Hoe werken ze samen en hoe wordt de validatie precies gedaan? Wat zijn de resultaten?

Install amesim and write a chapter about how the dataset is generated. How is the amesim model defined etc. 

\section{The First Topic of this Chapter}
\subsection{Item 1}
\subsubsection{Sub-item 1}


\subsubsection{Sub-item 2}


\subsection{Item 2}


\section{The Second Topic}


\section{Conclusion}

%%% Local Variables: 
%%% mode: latex
%%% TeX-master: "thesis"
%%% End: 

\chapter{Conclusion}
\label{cha:conclusion}
The final chapter contains the overall conclusion. It also contains
suggestions for future work and industrial applications.



%%% Local Variables: 
%%% mode: latex
%%% TeX-master: "thesis"
%%% End: 


% Indien er bijlagen zijn:
\appendixpage*          % indien gewenst
\appendix
\chapter{The First Appendix}
\label{app:A}
Appendices hold useful data which is not essential to understand the work
done in the master's thesis. An example is a (program) source.
An appendix can also have sections as well as figures and references\cite{h2g2}.

\section{More Lorem}


\subsection{Lorem 15--17}


\subsection{Lorem 18--19}


\section{Lorem 51}

%%% Local Variables: 
%%% mode: latex
%%% TeX-master: "thesis"
%%% End: 

% ... en zo verder tot
\chapter{The Last Appendix}
\label{app:n}
Appendices are numbered with letters, but the sections and subsections use
arabic numerals, as can be seen below.

\section{Lorem 20-24}


\section{Lorem 25-27}


%%% Local Variables: 
%%% mode: latex
%%% TeX-master: "thesis"
%%% End: 


\backmatter
% Na de bijlagen plaatst men nog de bibliografie.
% Je kan de  standaard "abbrv" bibliografiestijl vervangen door een andere.
\bibliographystyle{abbrv}
\bibliography{references}

\end{document}

%%% Local Variables: 
%%% mode: latex
%%% TeX-master: t
%%% End: 
