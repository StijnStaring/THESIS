\chapter{Comfort definition and modelling}
\label{cha:1}

Dan komt de vraag wat is comfort precies? Literatuur studie...
Waarnaar kan men kijken als men het over comfort heeft. 
Lane change bekeken om comfort te valideren --> zeg dat er geen iso standaarden aanwezig zijn.
Hoe wordt een bestuurder gemoddelleerd --> dit wordt gedaan door een kansverdeling.
Waarom is entropie nuttig om deze bestuurder te kunnen bekijken? Doe hier meer ondezoek over en verantwoord het gebruik hiervan. Conclusie komt af met comfort parameters die verder worden gebruikt als features.
Bij de uitleg van de features en waarom er versnelling en acceleratie wordt gebruikt, basseer ook op paper 7 van hoofdpaper

Wat wordt er in de literatuur al gebruikt om comfort te modelleren en geef een overzicht.

Leg uit hoe komt aan entropy distribution komt --> kan beroep doen op ref 2 en 20 van het hoofdrapport (both are assuming an exponential distribution) (IMPORTANT)


Dit is de reden voor het gebruik van de maximum entropie distributie: 
	To learn observed behavior, we aim to model the distribution
	that underlies the empirical sample trajectories.
	Following Abbeel and Ng [1], we aim to find a model that
	induces distributions that match, in expectation, the feature
	values fD of the empirical trajectories D, yielding
	Ep(x)[f (x)] = fD =
	1
	jDj
	X
	xk2D
	f (xk): (1)
	In general, however, there is not a unique distribution that
	matches the features. Ziebart et al. [24] resolve this ambiguity
	by applying the principle of maximum entropy [10], which
	states that the distribution with the highest entropy represents
	the given information best since it does not favor any
	particular outcome besides the observed constraints. (this is the least baised distribution)
	Modelling expected featueres by hybrid monte carlo method: https://reader.elsevier.com/reader/sd/pii/037026938791197X?token=71567B7640C402F2FF578E34E3BB7914CA14E4A1A29DE88D9534D96C9305E13DA52D0424FF1475A822FCE784725196D3
%	ref: C:\Users\t2vosx\OneDrive\Documenten\Leuven\Thesis\References2\Citations on main %paper\Learning to Predict Trajectories of Cooperatively Navigating Agents.pdf>




\section{The First Topic of the Chapter}
First comes the introduction to this topic.


\subsection{An item}
Please don't abuse enumerations: short enumerations shouldn't use
``\verb|itemize|'' or ``\texttt{enumerate}'' environments.
So \emph{never write}: 
\begin{quote}
  The Eiffel tower has three floors:
  \begin{itemize}
  \item the first one;
  \item the second one;
  \item the third one.
  \end{itemize}
\end{quote}
But write:
\begin{quote}
  The Eiffel tower has three floors: the first one, the second one, and the
  third one.
\end{quote}

\section{A Second Topic}


\subsection{Another item}


\section{Conclusion}
The final section of the chapter gives an overview of the important results
of this chapter. This implies that the introductory chapter and the
concluding chapter don't need a conclusion.



%%% Local Variables: 
%%% mode: latex
%%% TeX-master: "thesis"
%%% End: 
